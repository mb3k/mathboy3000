\documentclass[11pt]{article}

\usepackage[margin=1.0in]{geometry}
\usepackage{tabularx}
\usepackage{hyperref}
\hypersetup{
    colorlinks=true,
    linkcolor=blue,      
    urlcolor=cyan,
}

\urlstyle{same}

\newcommand{\code}[1]{\texttt{#1}}
\newcolumntype{L}[1]{>{\raggedright\arraybackslash}p{#1}}
\newcolumntype{C}[1]{>{\centering\arraybackslash}p{#1}}
\newcolumntype{R}[1]{>{\raggedleft\arraybackslash}p{#1}}

\begin{document}

\title{\textbf{Expression Calculator and Equation Solver} \\
    \url{https://github.com/stepheniskander/csc380project}}
\author{Nicholas Esposito \\ nesposi3@oswego.edu \and Stephen Iskander \\ siskande@oswego.edu}

\maketitle

\section{Project Description}
Our project is an expression calculator and equation solver.
The user can input mathematical expressions (e.g. \code{4*2+1}) to be evaluated to a numerical answer.
Expressions can also contain matrices or calculus operations such as integrals and derivatives.
If an expression contains one or more variables it will be simplified in terms of the variables (e.g. \code{(1*2)*x+(4-7) -> 2x-3}).
If the user inputs an equation containing a single variable, the program will solve for that variable.
The user can also input a system of equations that the program will solve if possible.

\par The program uses JavaFX for its user interface.
The main window is a simple command-prompt-like interface that the user types expressions or equations into.
Expressions are typed into a text field at the bottom of the window, and the expression as well as its evaluated answer are displayed in a text area above the input field.
The user can see their history in the output area so they can easily copy and paste previous expressions.
An additional window allows the user to graphically build matrices that will be inserted into the input field.
Systems of equations are handled on a separate tab to simplify inputting multiple equations.

\subsection{Stretch Goals}
The user can also use the program to graph functions.
Graphs are located on a separate tab.
The user inputs functions into an expandable list of functions and can choose what functions are visible on the graph.
A graph of all the visible functions is displayed next to the list of functions.
The window size and scale of the graph can be set manually or the program can attempt to set appropriate values.

\newpage

\section{System Requirements}

\begin{center}
\begin{tabular}{|L{.5in}|R{.5in}|L{5in}|}
\hline
\multicolumn{1}{|c|}{\textbf{Identifier}} & \multicolumn{1}{c|}{\textbf{Priority}} & \multicolumn{1}{c|}{\textbf{Description}} \\ \hline
REQ1  & 10 & Parsing user input into an expression. \\ \hline
REQ2  & 10 & Provide an interface for user to input expressions and see result.\\ \hline
REQ3  & 10 & Evaluate expression parse tree into numerical answer. \\ \hline
REQ4  & 9  & Evaluate an expression containing variables in terms of variables. \\ \hline
REQ5  & 8  & Solve an equation with a single variable. \\ \hline
REQ6  & 8  & Evaluate calculus operations numerically. \\ \hline
REQ7  & 7  & Evaluate calculus operations symbolically. \\ \hline 
REQ8  & 7  & Parse a matrix entered in text format ([[x y z] [a b c]]) \\ \hline
REQ9  & 7  & Perform matrix multiplication symbolically (i.e. each element is an expression). \\ \hline
REQ10 & 6  & Perform LU factorization on a matrix in order to solve a system of equations. \\ \hline
REQ11 & 5  & Provide a UI to simplify input of a system of equations. \\ \hline
REQ12 & 4  & Provide a graphical interface for building matrices. \\ \hline
REQ13 & 3  & Graph functions. \\ \hline
\end{tabular}
\end{center}

\newpage

\section{User Stories}

\begin{center}
\begin{tabular}{|L{.5in}|R{.5in}|L{5in}|}
\hline
\multicolumn{1}{|c|}{\textbf{Identifier}} & \multicolumn{1}{c|}{\textbf{Size}} & \multicolumn{1}{c|}{\textbf{Description}} \\ \hline
ST-1 & 3 pts & As a user, I can evaluate an expression by typing in the input field. \\ \hline
ST-2 & 6 pts & As a user, I can get a solution for an equation with a single variable. \\ \hline
ST-3 & 4 pts & As a user, I can evaluate an expression containing calculus operations numerically. \\ \hline
ST-4 & 6 pts & As a user, I can evaluate an expression containing calculus operations numerically. \\ \hline
ST-3 & 4 pts & As a user, I can input a matrix in a text format. \\ \hline
ST-4 & 5 pts & As a user, I can perform matrix multiplication. \\ \hline
ST-5 & 6 pts & As a user, I can perform matrix multiplication in terms of variable matrix elements. \\ \hline
ST-6 & 4 pts & As a user, I can use a graphical interface to simplify creating a matrix. \\ \hline
ST-7 & 7 pts & As a user, I can solve a system of equations. \\ \hline
ST-8 & 4 pts & As a user, I can use a graphical UI to simplify entering a system of equations. \\ \hline
ST-9 & 9 pts & As a user, I can graph a function. \\ \hline
\end{tabular}    
\end{center}

\newpage

\section{Use Cases}

\centering
\begin{tabular}{p{1.5in}p{5in}}
\hline
\textbf{Use Case UC-\#}     & \textbf{Name} \\ \hline
Related Requirements: & List of the requirements that are addressed by this use case \\
Initiating Actor:     & Actor who initiates interaction with the system to accomplish a goal \\
Actor's Goal:          & Informal description of the initiating actor’s goal \\
Preconditions:         & What is assumed about the state of the system before the interaction starts \\
Postconditions:        & What are the results after the goal is achieved or abandoned; i.e., what must be true about the system at the time the execution of this use case is completed
\end{tabular}

\begin{tabular}{p{.25in}p{.25in}p{5.8in}}
\multicolumn{3}{l}{Flow of Events for Main Success Scenario:} \\
$\rightarrow$ & 1. & The initiating actor delivers an action or stimulus to the system (the arrow indicates the direction of interaction, to- or from the system)
\end{tabular}

\begin{tabular}{p{.25in}p{.25in}p{5.8in}}
\multicolumn{3}{l}{Flow of Events for Extensions:} \\
$\rightarrow$ & 1a. & For example, the actor enters invalid data.
\end{tabular}

\end{document}
